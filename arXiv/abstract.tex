\lettrine[lines=3]{W}{e} are at the dawn of a data-driven era in astrophysics and cosmology. A large number of ongoing and forthcoming experiments combined with an increasingly open approach to data availability offer great potential in unlocking some of the deepest mysteries of the Universe. Among these is understanding the nature of dark matter (DM)---one of the major unsolved problems in particle physics. Characterizing DM through its astrophysical signatures will require a robust understanding of its distribution in the sky and the use of novel statistical methods. 

The first part of this thesis describes the implementation of a novel statistical technique which leverages the ``clumpiness'' of photons originating from point sources (PSs) to derive the properties of PS populations hidden in astrophysical datasets. This is applied to data from the \emph{Fermi} satellite at high latitudes ($|b|\geq 30^\circ$) to characterize the contribution of PSs of extragalactic origin. We find that the majority of extragalactic gamma-ray emission can be ascribed to unresolved PSs having properties consistent with known sources such as active galactic nuclei. This leaves considerably less room for significant dark matter contribution.

The second part of this thesis poses the question: ``what is the best way to look for annihilating dark matter in extragalactic sources?'' and attempts to answer it by constructing a pipeline to robustly map out the distribution of dark matter outside the Milky Way using galaxy group catalogs. This framework is then applied to \emph{Fermi} data and existing group catalogs to search for annihilating dark matter in extragalactic galaxies and clusters.